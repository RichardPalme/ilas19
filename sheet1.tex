\documentclass{article}
\usepackage[margin=0.7in]{geometry}
\usepackage[utf8]{inputenc}
\usepackage{amsmath,amssymb,amsfonts,amsthm}
\usepackage{mathtools}
\usepackage{algorithm}
\usepackage{algpseudocode}

\DeclarePairedDelimiter\ceil{\lceil}{\rceil}
\DeclarePairedDelimiter\floor{\lfloor}{\rfloor}

\newcommand{\pluseq}{\mathrel{+}=}
\newcommand{\minuseq}{\mathrel{-}=}

\newenvironment{claim}[1]{\par\noindent\underline{Claim:}\space#1}{}
\newenvironment{claimproof}[1]{\par\noindent\underline{Proof:}\space#1}{\hfill $\blacksquare$}

\setlength{\parindent}{0pt}


%%%%%%%%%%%%%%%%%%%%%%%%%%%%%%%%%%%%%%%%%%%%%%%%%%%%%%%%%%%%%%%%%%%%%%%%%%%%%%%
\begin{document}

\title{Intelligent Learning and Analysis Systems SS19, Exercise Sheet 1}
\author{Andreas Hene, Niklas Mertens, Richard Palme}
\date{\today}
\maketitle

%%%%%%%%%%%%%%%%%%%%%%%%%%%%%%%%%%%%%%%%%%%%%%%%%%%%%%%%%%%%%%%%%%%%%%%%%%%%%%%
\section*{1. Finding Two Missing Items}



%%%%%%%%%%%%%%%%%%%%%%%%%%%%%%%%%%%%%%%%%%%%%%%%%%%%%%%%%%%%%%%%%%%%%%%%%%%%%%%
\section*{2. Identifying the Majority}

Let $\sigma = \langle a_1, \hdots, a_m \rangle$ and $a_i \in [n]$. \\

\begin{algorithm}
\caption{Algorithm for finding the majority of $\sigma$, if it exists}
\begin{algorithmic}[1]

\State $c \gets 0$
    \ForAll{$i \in [m]$}
    \If{$c=0$}
        \State$s \gets a_i$
        \State$c \gets 1$
    \EndIf
    \If{$s=a_i$}
        \State$c \pluseq 1$
    \Else
        \State$c \minuseq 1$
    \EndIf
\EndFor
\State\textbf{return} $s$

\end{algorithmic}
\end{algorithm}

Let $A(\sigma)$ be the output of the algorithm. \\

\begin{claim}
$A(\sigma)$ is the majority of $\sigma$, if it exists.
\end{claim}

\begin{claimproof}
Let $k$ be the number of times $c$ is set to 0 in the for loop. We prove the claim by induction over $k$. \\

Base case:\\
Let $k=0$. Since $c$ is never set to $0$ during the loop,
we know that $s$ never changes after iteration 1, i.e. $A(\sigma) = a_1$.
Also, $c$ increases at least one time more often than it decreases,
so $A(\sigma) = a_1$ has to be the majority of $\sigma$. \\

Step case:\\
Let $k>0$. Suppose the claim is true for all streams for which $c$ is set to 0 at most $k-1$ times in the for loop.
Let $d$ be the majority of $\sigma$. Let $j$ be the first iteration where $c$ is set to 0.
Side note: When c is set to 0 in an iteration $j'$, $j'$ has to be even.
$d$ can occur at most $\frac{j}{2}$ times in $\tau \coloneqq \langle a_1, \hdots, a_j \rangle$,
otherwise $d$ would be the majority of $\tau$ and $c$ couldn't be set to 0 in iteration $j$. Consequently $d$ will occur at least
\[
\floor*{\frac{m}{2}} + 1 - \frac{j}{2} = \floor*{\frac{m-j}{2}} + 1
\]
times in $\sigma' = \langle a_{j+1}, \hdots, a_m \rangle$.
Because $\sigma'$ is a stream of length $m-j$, $d$ is the majority of $\sigma'$.
Since the number of zeros in $\sigma'$ is smaller than $k$,
we can apply the inductive hypothesis to see that $A(\sigma') = d$.
    Since in iteration $j$ of computing $A(\sigma)$, $c$ is set to 0, applying the algorithm to $\sigma$ is the same as
applying it to $\tau$ and then to $\sigma'$. Hence, $A(\sigma) = d$.

\end{claimproof}


%%%%%%%%%%%%%%%%%%%%%%%%%%%%%%%%%%%%%%%%%%%%%%%%%%%%%%%%%%%%%%%%%%%%%%%%%%%%%%%
\section*{3. Reservoir Sampling I.}



%%%%%%%%%%%%%%%%%%%%%%%%%%%%%%%%%%%%%%%%%%%%%%%%%%%%%%%%%%%%%%%%%%%%%%%%%%%%%%%
\section*{4. Reservoir Sampling II.}



%%%%%%%%%%%%%%%%%%%%%%%%%%%%%%%%%%%%%%%%%%%%%%%%%%%%%%%%%%%%%%%%%%%%%%%%%%%%%%%
\section*{5. The $\phi$-HH Problem: Lower Bounds}


\end{document}
